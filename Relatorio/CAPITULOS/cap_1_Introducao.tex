\chapter{INTRODUÇÃO} \label{cap_1_Introducao}

Nessa primeira prática, a utilização da \emph{FPGA} foi para descobrir uma senha de 16 valores hexadecimais, que foi fixada pelo grupo. Para descobrir essa senha tem-se várias tentativas.


O objetivo do trabalho é a utilização da plataforma quartus e, por meio de portas lógicas, flip-flops e outros recursos dela em diagrama de blocos, elaborar o sistema pedido na primeira prática. Para atingir esse objetivo, é necessário: a elaboração de contadores MOD10 e MOD2, a elaboração de armazenadores de valores hexadecimais de 4 bits e conexão entre as tentativas de adivinhar a senha com certos leds e buzzer, com devidas frequências.


Este relatório estará dividido em 5 seções. A primeira é o desenvolvimento, no qual será explicado as etapas em que foi feito so processo com suas devidas explicações. A segunda é o manual de operações, no qual será explicado todo o funcionamento do hardware projetado na fpga. A terceira é os resulatdos, que terá, em ordem, o que foi obtido como resposta ao longo do desenvolvimento do projeto. A quarta é a discussão dos resultados, que abordará do porque desses resultados terem sido atingidos. A última é a conclusão que irá discutir, de forma resumida, se os resultados obtidos foram os solicitados no projeto.  


